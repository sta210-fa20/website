% Options for packages loaded elsewhere
\PassOptionsToPackage{unicode}{hyperref}
\PassOptionsToPackage{hyphens}{url}
%
\documentclass[
]{article}
\usepackage{lmodern}
\usepackage{amssymb,amsmath}
\usepackage{ifxetex,ifluatex}
\ifnum 0\ifxetex 1\fi\ifluatex 1\fi=0 % if pdftex
  \usepackage[T1]{fontenc}
  \usepackage[utf8]{inputenc}
  \usepackage{textcomp} % provide euro and other symbols
\else % if luatex or xetex
  \usepackage{unicode-math}
  \defaultfontfeatures{Scale=MatchLowercase}
  \defaultfontfeatures[\rmfamily]{Ligatures=TeX,Scale=1}
\fi
% Use upquote if available, for straight quotes in verbatim environments
\IfFileExists{upquote.sty}{\usepackage{upquote}}{}
\IfFileExists{microtype.sty}{% use microtype if available
  \usepackage[]{microtype}
  \UseMicrotypeSet[protrusion]{basicmath} % disable protrusion for tt fonts
}{}
\makeatletter
\@ifundefined{KOMAClassName}{% if non-KOMA class
  \IfFileExists{parskip.sty}{%
    \usepackage{parskip}
  }{% else
    \setlength{\parindent}{0pt}
    \setlength{\parskip}{6pt plus 2pt minus 1pt}}
}{% if KOMA class
  \KOMAoptions{parskip=half}}
\makeatother
\usepackage{xcolor}
\IfFileExists{xurl.sty}{\usepackage{xurl}}{} % add URL line breaks if available
\IfFileExists{bookmark.sty}{\usepackage{bookmark}}{\usepackage{hyperref}}
\hypersetup{
  pdftitle={Syllabus},
  hidelinks,
  pdfcreator={LaTeX via pandoc}}
\urlstyle{same} % disable monospaced font for URLs
\usepackage[margin=1in]{geometry}
\usepackage{longtable,booktabs}
% Correct order of tables after \paragraph or \subparagraph
\usepackage{etoolbox}
\makeatletter
\patchcmd\longtable{\par}{\if@noskipsec\mbox{}\fi\par}{}{}
\makeatother
% Allow footnotes in longtable head/foot
\IfFileExists{footnotehyper.sty}{\usepackage{footnotehyper}}{\usepackage{footnote}}
\makesavenoteenv{longtable}
\usepackage{graphicx,grffile}
\makeatletter
\def\maxwidth{\ifdim\Gin@nat@width>\linewidth\linewidth\else\Gin@nat@width\fi}
\def\maxheight{\ifdim\Gin@nat@height>\textheight\textheight\else\Gin@nat@height\fi}
\makeatother
% Scale images if necessary, so that they will not overflow the page
% margins by default, and it is still possible to overwrite the defaults
% using explicit options in \includegraphics[width, height, ...]{}
\setkeys{Gin}{width=\maxwidth,height=\maxheight,keepaspectratio}
% Set default figure placement to htbp
\makeatletter
\def\fps@figure{htbp}
\makeatother
\setlength{\emergencystretch}{3em} % prevent overfull lines
\providecommand{\tightlist}{%
  \setlength{\itemsep}{0pt}\setlength{\parskip}{0pt}}
\setcounter{secnumdepth}{-\maxdimen} % remove section numbering

\title{Syllabus}
\author{}
\date{\vspace{-2.5em}}

\begin{document}
\maketitle

\textbf{\href{sta210-sp21-syllabus.pdf}{Click here} to download a PDF
copy of the syllabus.}

\hypertarget{course-learning-objectives}{%
\subsection{Course learning
objectives}\label{course-learning-objectives}}

By the end of the semester, you will be able to\ldots{}

\begin{itemize}
\tightlist
\item
  Analyze real-world data to answer questions about multivariable
  relationships.
\item
  Fit and evaluate linear and logistic regression models.
\item
  Assess whether a proposed model is appropriate and describe its
  limitations.
\item
  Use R Markdown to write reproducible reports and GitHub for version
  control and collaboration.
\item
  Communicate results from statistical analyses to a general audience.
\end{itemize}

\hypertarget{course-community}{%
\subsection{Course community}\label{course-community}}

\hypertarget{duke-compact-community-standard}{%
\subsubsection{Duke Compact \& Community
Standard}\label{duke-compact-community-standard}}

\textbf{The Duke Compact}

The Duke Compact recognizes our shared responsibility for our collective
health and well- being. Please be reminded that by signing your name to
this pledge, you have acknowledged that you understand the conditions
for being on campus. These include complying with university, state, and
local requirements and acting to protect yourself and those around you.
For complete language and updated policies, please visit
\href{https://returnto.duke.edu/compact/}{returnto.duke.edu/compact/}.

\textbf{Duke Community Standard}

All students, whether residing on campus or learning remotely, must
adhere to the
\href{https://trinity.duke.edu/undergraduate/academic-policies/community-standard-student-conduct}{Duke
Community Standard (DCS)}: Duke University is a community dedicated to
scholarship, leadership, and service and to the principles of honesty,
fairness, and accountability. Citizens of this community commit to
reflect upon these principles in all academic and non- academic
endeavors, and to protect and promote a culture of integrity.

To uphold the Duke Community Standard:

\emph{Students affirm their commitment to uphold the values of the Duke
University community by signing a pledge that states:}

\begin{quote}
I will not lie, cheat, or steal in my academic endeavors;
\end{quote}

\begin{quote}
I will conduct myself honorably in all my endeavors;
\end{quote}

\begin{quote}
I will act if the Standard is compromised
\end{quote}

Regardless of course delivery format, it is the responsibility of all
students to understand and follow Duke policies regarding academic
integrity, including doing one's own work, following proper citation of
sources, and adhering to guidance around group work projects. Ignoring
these requirements is a violation of the Duke Community Standard. If you
have any questions about how to follow these requirements, please
contact \href{mailto:jeanna.mccullers@duke.edu}{Jeanna McCullers},
Director of the Office of Student Conduct.

\hypertarget{inclusive-community}{%
\subsubsection{Inclusive community}\label{inclusive-community}}

It is my intent that students from diverse backgrounds and perspectives
be well-served by this course, that students' learning needs be
addressed both in and out of class, and that the diversity that the
students bring to this class be viewed as a resource, strength and
benefit. It is my intent to present materials and activities that are
respectful of diversity and in alignment with
\href{https://provost.duke.edu/initiatives/commitment-to-diversity-and-inclusion}{Duke's
Commitment to Diversity and Inclusion}. Your suggestions are encouraged
and appreciated. Please let me know ways to improve the effectiveness of
the course for you personally, or for other students or student groups.

Furthermore, I strive to create a learning environment that supports a
diversity of thoughts, perspectives and experiences. To help accomplish
this:

\begin{itemize}
\tightlist
\item
  If you feel like your performance in the class is being impacted by
  your experiences outside of class, please don't hesitate to come and
  talk with me. If you prefer to speak with someone outside of the
  course, your academic dean is an excellent resource.
\item
  If something was said or done as part of this class that makes you
  uncomfortable, please let me know.
\end{itemize}

\hypertarget{accessibility}{%
\subsection{Accessibility}\label{accessibility}}

In addition to accessibility issues experienced during the typical
academic year, I recognize that remote learning may present additional
challenges. Students may be experiencing unreliable wi-fi, lack of
access to quiet study spaces, varied time-zones, or additional
responsibilities while studying at home. If you are experiencing these
or other difficulties, please contact me to discuss possible
accommodations.

\hypertarget{academic-accommodations}{%
\subsubsection{Academic accommodations}\label{academic-accommodations}}

The \href{https://access.duke.edu/students}{Student Disability Access
Office (SDAO)} will continue to be available to ensure that students are
able to engage with their courses and related assignments. Students
should be in touch with the Student Disability Access Office to
\href{https://access.duke.edu/requests}{request or update}
accommodations under these circumstances. Zoom has the ability to
provide live closed captioning. If you are not seeing this, and but
would like to see this feature, please reach out to your instructor for
assistance.

\hypertarget{course-materials-communication}{%
\subsection{Course materials \&
communication}\label{course-materials-communication}}

\hypertarget{materials}{%
\subsubsection{Materials}\label{materials}}

All lecture notes, assignment instructions, up-to-date schedule, and
other course materials may be found on the course website,
\url{https://sta210-sp21.netlify.app/}, specifically on the
\href{https://sta210-sp21.netlify.app/schedule/}{\textbf{Schedule}}
page.

See the weekly pages for a detailed outline of each week's materials,
assignments, videos, and activities.

We will use RStudio and GitHub for computing in the class. You may use
RStudio through the STA 210 - Regression Analysis
\href{https://vm-manage.oit.duke.edu/containers}{Docker container}. See
the \href{/resources/}{Resources} page for resources to help you get
started with RStudio and GitHub.

\hypertarget{communication}{%
\subsubsection{Communication}\label{communication}}

Regular announcements will be sent to the class through Sakai. Please
check your email or check the \textbf{Announcements} tab in Sakai
regularly to ensure you have up-to-date information about the course.

\textbf{Online communication}

If you have general questions about the course logistics, content, or
assignments, you may post them on the online discussion platform. Note
that this forum should only be used for questions or comments that may
be viewed by the entire class.

\textbf{Email}

If there is a question that's not appropriate for the online discussion
platform, you are welcome to email me directly. \textbf{If you email me,
please include ``STA 210'' in the subject line.} Barring extenuating
circumstances, I will respond to STA 210 emails within 48 hours Monday -
Friday.

\hypertarget{activities-assessment}{%
\subsection{Activities \& Assessment}\label{activities-assessment}}

The following activities and assessments will help you successfully
achieve the course learning objectives. By experiencing the course
content in different ways, you will not only gain a better understanding
of regression analysis, but you will also get experiences that can guide
you as you apply what you've learned in future academic and professional
settings.

\hypertarget{a-week-in-sta210}{%
\subsubsection{A week in STA210}\label{a-week-in-sta210}}

\begin{longtable}[]{@{}ll@{}}
\toprule
\textbf{Day} & \textbf{Activity}\tabularnewline
\midrule
\endhead
\textbf{Monday} & Attend lab\tabularnewline
& Watch lecture content video for Tuesday\tabularnewline
\textbf{Tuesday} & Attend live lecture session 10:15a -
11:30a\tabularnewline
\textbf{Wednesday} & Tuesday's application exercise due\tabularnewline
& Watch lecture content video for Thursday\tabularnewline
\textbf{Thursday} & Attend live lecture session 10:15a -
11:30a\tabularnewline
\textbf{Friday} & Thursday's application exercise due\tabularnewline
\bottomrule
\end{longtable}

\hypertarget{lectures}{%
\subsubsection{Lectures}\label{lectures}}

Lectures will have two components:

\begin{itemize}
\item
  \textbf{Lecture content videos}: These are pre-recorded videos that
  contain the content. You can think of these as a ``video textbook''.
  You should watch the content videos before we meet for the live
  sessions.
\item
  \textbf{Live lecture sessions}: These sessions will be on Zoom during
  the scheduled class time. During this time, we will answer questions
  from the live lecture videos and work through application exercises
  (AE) to apply what you learned in the content videos.
\end{itemize}

\hypertarget{labs}{%
\subsubsection{Labs}\label{labs}}

In labs, you will apply the concepts discussed in lecture to various
data analysis scenarios, with a focus on the computation. Most lab
assignments will be performed in teams, and all team members are
expected to contribute equally to the completion of each assignment. You
are expected to use the team's Git repository in the
\href{https://github.com/sta210-sp21}{STA 210 GitHub organization} as
the central platform for collaboration. Commits to this repository will
be used as a metric of each team member's relative contribution for each
lab, and you will also be asked to evaluate your team members'
engagement periodically during the semester. Lab assignments will be
completed using R Markdown, correspond to an appropriate GitHub
repository, and submitted to Gradescope.

\textbf{Sections 01L - 03L}: If you are in one of these sections, you
are expected to attend lab during the scheduled time. You will spend
most time working with your team to complete that week's assignment.
There will be teaching assistants available if you have questions as you
work.

\textbf{Section 04L:} Though you do not have a scheduled lab time, you
will be assigned a team to work on the weekly lab assignment. There will
be a short video to introduce each lab, and the teaching team will
provide lab hours specifically for asynchronous lab students.

Your lowest lab grade will be dropped at the end of the semester.

\hypertarget{homework}{%
\subsubsection{Homework}\label{homework}}

In most homework assignments, you will apply what you've learned during
lecture and lab to complete data analysis tasks. You may discuss
homework assignments with other students; however, homework should be
completed and submitted individually. Homework must be typed up using R
Markdown and GitHub and submitted to Gradescope.

One homework assignment will be dedicated towards engaging with
statistics outside of the classroom by attending statistics-related
talks, participating in a data analysis competition, listening to
related podcasts, or similar activities. More details will be provided
during the semester.

Your lowest homework grade will be dropped at the end of the semester.

\hypertarget{quizzes}{%
\subsubsection{Quizzes}\label{quizzes}}

The quizzes are an opportunity to assess the knowledge and skills you've
learned. They will include both the conceptual and mathematical aspects
of regression. More details will be provided before each quiz.

Your lowest quiz grade will be down-weighted in the calculation of your
final quiz average.

\hypertarget{final-project}{%
\subsubsection{Final Project}\label{final-project}}

The purpose of the project is to apply what you've learned throughout
the semester to explore an interesting data-based research question
using regression. The project will be completed with your lab teams, and
each team will present their work in video and in writing during the
final exam period. More information about the project will be provided
later in the semester.

\hypertarget{participation}{%
\subsubsection{Participation}\label{participation}}

The participation component of the grade will be based on two
components:

\begin{itemize}
\item
  \textbf{Participating in lecture.} This includes watching the lecture
  content videos and completing the Application Exercises (AEs) from the
  corresponding live lecture session. AEs are due on GitHub within one
  week of the live lecture. AEs will graded based on making a good-faith
  effort on the activity. You will receive full credit for this portion
  of the participation grade if you watch the lecture content video and
  complete the AE in a timely manner for at least 80\% of the lectures.
\item
  \textbf{Teamwork} This will be based periodic team feedback and
  commits to team labs and the final program.
\end{itemize}

\hypertarget{grading}{%
\subsection{Grading}\label{grading}}

The final course grade will be calculated as follows:

\begin{longtable}[]{@{}ll@{}}
\toprule
{} Category & {} Percentage\tabularnewline
\midrule
\endhead
Quizzes & 30\%\tabularnewline
Homework & 30\%\tabularnewline
Final Project & 20\%\tabularnewline
Labs & 15\%\tabularnewline
Participation & 5\%\tabularnewline
\bottomrule
\end{longtable}

The final letter grade will be determined based on the following
thresholds:

\begin{longtable}[]{@{}ll@{}}
\toprule
{} Letter Grade & {} Final Course Grade\tabularnewline
\midrule
\endhead
A & \textgreater= 93\tabularnewline
A- & 90 - 92.99\tabularnewline
B+ & 87 - 89.99\tabularnewline
B & 83 - 86.99\tabularnewline
B- & 80 - 82.99\tabularnewline
C+ & 77 - 79.99\tabularnewline
C & 73 - 76.99\tabularnewline
C- & 70 - 72.99\tabularnewline
D+ & 67 - 69.99\tabularnewline
D & 63 - 66.99\tabularnewline
D- & 60 - 62.99\tabularnewline
F & \textless{} 60\tabularnewline
\bottomrule
\end{longtable}

\hypertarget{asking-for-help}{%
\subsection{Asking for help}\label{asking-for-help}}

\begin{itemize}
\tightlist
\item
  \textbf{If you have a question during lecture or lab, please to ask
  it!} There are likely other students with the same question, so it is
  a learning opportunity for everyone.
\item
  The teaching team is here to help you be successful in the course. A
  lot of questions are most effectively answered through discussion
  rather than email, so office hours are a valuable resource. They are
  your time to ask questions about course content and assignments. The
  office hours schedule is posted on the \href{/home/}{homepage}.
\item
  Outside of class and office hours, any general questions about course
  content or assignments should be posted on the online discussion
  platform. There is a chance another student has already asked a
  similar question, so please check the other posts on online discussion
  platform before adding a new question. If you know the answer to a
  question posted, I encourage you to respond!
\end{itemize}

See the \href{/resources/}{Resources} tab for information about
additional resources for the course.

\hypertarget{academic-conduct}{%
\subsection{Academic conduct}\label{academic-conduct}}

\textbf{TL;DR: Don't cheat!}

Please abide by the following as you work on assignments in this course:

\begin{itemize}
\item
  On individual assignments such as homework, you may discuss the
  assignment with other students; however, you may not directly share
  your code or write up with any other student. On team assignments such
  as labs, you may discuss the assignment with other teams; however, you
  may not directly share your team's code or write up with another team.
  Unauthorized sharing of code or a write up is considered an instance
  of academic misconduct for all students involved.
\item
  \textbf{Reusing code}: Unless explicitly stated otherwise, you may
  make use of online resources (e.g.~StackOverflow) for coding examples
  on assignments. If you directly use code from an outside source (or
  use it as inspiration), you must explicitly cite where you obtained
  the code. Any recycled code that is discovered and is not explicitly
  cited will be treated as plagiarism and an instance of academic
  misconduct.
\item
  You may not discuss or otherwise work with anyone on quizzes.
  Unauthorized collaboration or use of unauthorized materials is
  considered an instance of academic misconduct for all students
  involved. More specific details about standards for academic conduct
  will be provided with the quiz instructions.
\end{itemize}

Any violations in the standards for academic conduct as outlined in the
\href{https://studentaffairs.duke.edu/conduct/about-us/duke-community-standard}{Duke
Community Standard} and those specific to this course will automatically
result in a 0 for the assignment and will be reported to the
\href{https://studentaffairs.duke.edu/conduct}{Office of Student
Conduct} for appropriate additional action.

\hypertarget{late-work-extensions}{%
\subsection{Late work \& extensions}\label{late-work-extensions}}

The due dates for assignments are there to help you keep up with the
course material and to ensure the teaching team can provide feedback
within a timely manner. We understand that things come up periodically
that could make it difficult to submit an assignment by the deadline.
Note that the lowest homework and lab assignment will be dropped and the
lowest quiz grade will be down-weighted to accommodate such
circumstances.

\hypertarget{late-work-policy}{%
\subsubsection{Late work policy}\label{late-work-policy}}

\begin{itemize}
\tightlist
\item
  Homework and labs may be submitted up to 3 days late with a 5\%
  penalty each day.
\item
  There is no late work accepted for application exercises or quizzes.
\item
  The late work policy for the project will be provided with the project
  instructions.
\end{itemize}

\hypertarget{waiver-for-extenuating-circumstances}{%
\subsubsection{Waiver for extenuating
circumstances}\label{waiver-for-extenuating-circumstances}}

If there are circumstances that prevent you from completing a lab or
homework assignment by the stated due date, please email Professor
Tackett before the deadline, to waive the late penalty. In your email,
you only need to request the waiver; you do not need to provide
explanation. This waiver may only be used once, so only use it for a
truly extenuating circumstance.

If there are circumstances that are having a longer-term impact on your
academic performance, please notify your academic dean.

\hypertarget{regrade-requests}{%
\subsection{Regrade Requests}\label{regrade-requests}}

Regrade requests must be submitted on Gradescope within a week of when
an assignment is returned. Regrade requests will be considered if there
was an error in the grade calculation or if you feel a correct answer
was mistakenly marked as incorrect. Requests to dispute the number of
points deducted for an incorrect response will not be considered. Note
that by submitting a regrade request, the entire question may be
regraded and you may potentially lose points.

No grades will be changed after the final project presentations.

\hypertarget{additional-resources}{%
\subsection{Additional resources}\label{additional-resources}}

Please see the \href{../resources/}{Resources} page for additional
academic and wellness resources.

\hypertarget{important-dates}{%
\subsection{Important dates}\label{important-dates}}

You can find important dates on the
\href{https://registrar.duke.edu/spring-2021-academic-calendar}{Spring
2021 Academic Calendar}.

\end{document}
